\chapter{Further readings} \label{ch:readings}

In addition to our book, we have written an article explaining how to use the R version of the library:
\begin{quote}
Matteo Magnani, Luca Rossi and Davide Vega. Multiplex network analysis with R. Journal of Statistical Software. (forthcoming) 
\end{quote}
and various articles summarizing the state of the art on specific aspects of multilayer network analysis (at the time of writing), including overviews about community detection: 
\begin{quote}
Matteo Magnani, Obaida Hanteer, Roberto Interdonato, Luca Rossi and Andrea Tagarelli. Community Detection in Multiplex Networks. ACM Computing Surveys (forthcoming) 
\end{quote}
pre-processing:
\begin{quote}
Roberto Interdonato, Matteo Magnani, Diego Perna, Andrea Tagarelli and Davide Vega. Multilayer network simplification: approaches, models and methods. Computer Science Review, 36, Elsevier.
\end{quote}
layer comparison:
\begin{quote}
Piotr Brodka, Anna Chmiel, Matteo Magnani and Giancarlo Ragozini (2018). Quantifying layer similarity in multiplex networks: a systematic study. Royal Society Open Science, 5(8). 
\end{quote}
and diffusion/propagation:
\begin{quote}
Mostafa Salehi, Rajesh Sharma, Moreno Marzolla, Matteo Magnani, Payam Siyari, and Danilo Montesi (2015). Spreading Processes in Multilayer Networks. IEEE Transactions on Network Science and Engineering 2 (2): 65--83.
\end{quote}

The following survey papers on general multilayer networks can also be of interest as they provide broad overviews of the research on multilayer networks from different perspectives: 

\begin{quote}
Mikko Kivel\"a, Alexandre Arenas, Marc Barthelemy, James P. Gleeson, Yamir Moreno, and Mason A. Porter. 2014. Multilayer Networks. Physics and Society. Journal of Complex Networks 2 (3): 203--71. 
\end{quote}

\begin{quote}
Stefano Boccaletti, Ginestra Bianconi, Regino Criado, Charo I. Del Genio, Jes\'us G\'omez-Garde\~nes, Miguel Romance, Irene Sendi\~na-Nadal, Zhen Wang, and Massimiliano Zanin. 2014. The Structure and Dynamics of Multilayer Networks. Physics Reports 544 (1): 1--122.
\end{quote}