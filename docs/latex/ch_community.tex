\chapter{Community detection} \label{ch:comm}

The \code{community/} module provides three types of functionality: data structures to represent communities, community detection algorithms and evaluation functions.

\section{Data structures}

A Community Structure is a set of communities, and a Community is a set of elements. Communities are represented using a template class so that the type of elements adapts to the type of network. For example, in a community over networks its elements are vertices:
\begin{lstlisting}[style=c++]
auto com = make_unique<CommunityStructure<Network>>();

auto c1 = make_unique<Community<Network>>();
auto c2 = make_unique<Community<Network>>();

auto v1 = make_unique<Vertex>("v1");
auto v2 = make_unique<Vertex>("v2");

// Adding vertices to the community

c1->add(v1.get());
c1->add(v2.get());
c2->add(v2.get());

// Adding communities to the community structure

com->add(move(c1));
com->add(move(c2));
\end{lstlisting}

A community over a multilayer network has multilayer vertices as elements:
\begin{lstlisting}[style=c++]
auto ml_c = make_unique<Community<MultilayerNetwork>>();

// Creating multilayer vertices

auto n = make_unique<Network>("net");
auto ml_v1 = MLVertex(v1.get(), n.get());
auto ml_v2 = MLVertex(v2.get(), n.get());

// Adding multilayer vertices to the community

ml_c->add(ml_v1);
ml_c->add(ml_v2);
\end{lstlisting}

In general, one would not manually create communities, but obtain them using community detection algorithms.

\section{Algorithms}

The library provides two community detection algorithms for single networks:
\begin{lstlisting}[style=c++]
auto comm1 = label_propagation(net.get());
auto comm2 = louvain(net.get());
\end{lstlisting}
    
and five algorithms for multilayer networks.
\begin{lstlisting}[style=c++]
auto mlcomm1 = mlcpm(mlnet.get(), 3, 1);
auto mlcomm2 = abacus(mlnet.get(), 3, 1);
auto mlcomm3 = glouvain2(mlnet.get(), 1.0);
auto mlcomm4 = infomap(mlnet.get());
auto mlcomm5 = mlp(mlnet.get());
\end{lstlisting}
    
\section{Evaluation}

Two community structures can be compared using normalized mutual information (for partitioning communities) and omega index (for overlapping communities):
\begin{lstlisting}[style=c++]
nmi(comm1.get(), comm2.get(), order(net.get()));
omega_index(comm1.get(), comm2.get(), order(net.get()));
\end{lstlisting}

The library also provides a function to compute multilayer modularity:
\begin{lstlisting}[style=c++]
modularity(mlnet.get(), mlcomm3.get(), 1.0);
\end{lstlisting}

\section{I/O}

If one needs to compare communities found by an algorithm with some ground truth, then it can be useful to read communities from a file. The format for multilayer communities requires the name of the actor, the name of the layer and the community id, starting from 0.

\begin{lstlisting}[style=file]
vertex_name1,layer_name1,0
vertex_name2,layer_name1,0
% etc.
vertex_name1,layer_name2,1
% etc.
\end{lstlisting}

The \code{io/} module provides functions to read and write community files. Notice that the second parameter of the reading function is a multilayer network, so that the names of actors and layers can be matched with those in the network.
\begin{lstlisting}[style=c++]
read_multilayer_communities("com.txt", cml.get());

write_multilayer_communities(communities.get(), fname);
\end{lstlisting}