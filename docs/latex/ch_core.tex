\chapter{The core module}

The core module (\texttt{core/}) provides basic functions that are not specific of networks. This includes storing objects, handling attributes, managing exceptions, generating random numbers, reading CSV files, etc.

\section{Stores and attributes}

One main functionality provided by the \texttt{core} module is to store and retrieve objects and assign attributes to them. This is implemented in the \texttt{core/stores} and \texttt{core/attributes} sub-modules.

\subsection{Object stores}

The class \texttt{ObjectStore} is defined in \texttt{core/stores}. It can be used to store and retrieve objects with the following features:
\begin{enumerate}
\item defining a typedef key\_type (the type of the key used to identify the object in the store),
\item providing a const member function key() returning the key value for the object, and
\item inheriting from enable\_shared\_from\_this.
\end{enumerate}
Notice that the key is only used to retrieve the objects from the stores where they are indexed. Therefore, a key value is guaranteed to be unique only inside an Object Store. Different objects can have the same key. To check if two objects are the same, one should use the == operator (if provided by the class) or compare their pointers.

Object Stores are mainly used to store vertices and edges, but in the following example we use objects representing people, identified by their social security number (ssn) and also having a name. The social security number is also used as a key.
\begin{lstlisting}[style=c++]
class
Person :
    public enable_shared_from_this<Person>
{
    
  public:
    
    const string ssn;
    const string name;
    
    typedef string key_type;
    key_type key() const {return ssn;}
    
    Person(
        const string& ssn, 
        const string& name
    ) : ssn(ssn), name(name) {};
};
\end{lstlisting}
Now we can create a store and add, retrieve, and erase objects (that is, people in this case):
\begin{lstlisting}[style=c++]
ObjectStore<Person> store;

auto p1 = make_shared<Person>("0001", "Alice");
auto p2 = make_shared<Person>("0002", "Hatter");

store.add(p1.get());
store.add(p2.get());

store.size(); // 2
store.contains(p1.get()); // true
store.contains("0001"); // true
store.get("0002"); // p2
store.get_at_random(); // p1 or p2
store.at(0); // p1 or p2
store.index_of(p1.get()); // 0 or 1
store.erase(p1.get()); // true
store.erase("0001"); // false (already erased)
\end{lstlisting}
Notice that \texttt{store.at(0)} returns either p1 or p2 depending on the execution: Object Stores are sets, so  there is no fixed order for the objects they contain. However, while an Object Store is not modified the order is fixed.


\subsection{Attributes}

Attribute Stores, introduced in the next section, allow us to associate values to objects. The sub-module \texttt{core/attributes} provides basic functions to manipulate attributes. The file \texttt{AttributeType.hpp} defines the supported attribute types:
\begin{itemize}
    \item STRING (string)
    \item DOUBLE (double)
    \item INTEGER (int)
    \item TIME (Time -- alias for time\_t)
    \item TEXT (Text -- alias for string)
    \item STRINGSET (set<string>)
    \item DOUBLESET (set<double>)
    \item INTEGERSET (set<int>)
    \item TIMESET (set<Time>)
\end{itemize}

The first five are simple types, and we would typically store them wrapped into a \texttt{Value} object that allows us to represent null values. For example, a \texttt{Value<double>} has two fields: a flag \texttt{null} and a double field \texttt{value}. The value in \texttt{value} is only usable if the null flag is not set.

The reason why we have aliased time and text basic c++ types is that in the future we may decide to extend them with additional capabilities.

The file \texttt{conversion.hpp} contains various functions to manipulate attributes. In particular, it provides functions to create Time values:
\begin{itemize}
    \item \texttt{epoch\_to\_time}, taking as input the number of seconds since epoch (as int or string),
    \item \texttt{to\_time}, taking as input a string and the format specifying how to interpret the string. Valid formats are described in the documentation of the \texttt{get\_time} function in the standard library.
\end{itemize}
One example of format specifier (the default one) is:\\ \texttt{\%Y-\%m-\%d~\%H:\%M:\%S~\%z}.

\subsection{Attribute stores} \label{ch:core:attr}

Attribute Stores allow us to associate attribute values to objects.
\begin{lstlisting}[style=c++]
AttributeStore<Person> attr;
attr.add("A1", AttributeType::DOUBLE);
attr.add("A2", AttributeType::DOUBLESET);
\end{lstlisting}
The functions available to assign and retrieve attribute values depend on the type o attribute. Simple types provide set and get functions:
\begin{lstlisting}[style=c++]
attr.set_double(p1.get(), "A1", 3.14); // void
attr.get_double(p1.get(), "A1");    // Value<double>
                                    // null set to false
attr.get_double(p3.get(), "A1");    // Value<double>
                                    // null set to true
\end{lstlisting}
While set types provide add and get functions:
\begin{lstlisting}[style=c++]
attr.add_double(p2.get(), "A2", 3.14); // void
attr.get_doubles(p2.get(), "A2");   // set<double>
                                    // size = 1
attr.get_doubles(p3.get(), "A2");   // set<double>
                                    // size = 0
\end{lstlisting}

\section{Utilities}

The sub-module \texttt{core/utils} contains a variety of functions to manipulate strings (\texttt{string.hpp}), perform mathematical operations (\texttt{math.hpp}), generate random numbers (\texttt{random.hpp}). Some of these functions were implemented in the library because they were not available in all the implementations of the standard library on all the systems where the library is expected to work --- in particular, the R package depending on this library is tested against several systems and configurations. The intention is to use standard functions when possible, and some of these can be replaced with standard function in time when they become (broadly) available. In the following we show four classes that are used in various other modules.


\subsection{Counter.hpp}

A \code{Counter} allows us to count the occurrences of any object, either by increasing the count by 1 (\code{inc}) or by directly setting a value (\code{set}).
\begin{lstlisting}[style=c++]
Counter<char> c;
c.count('a');  // 0
c.inc('a');    // c.count('a') would now return 1
c.set('a', 3); // c.count('a') would now return 3
c.inc('a');    // c.count('a') would now return 4
c.max(); // one of the elements with highest count 
         // ('a' in this case)
\end{lstlisting}


\subsection{CSVReader.hpp}

CSV files can be read using a \code{CSVReader}. This works
for both Windows and Unix/Mac systems (handling the different ways to start a new line), and provides settings to trim white characters at the beginning and the end of each field, to specify different separators, and to indicate which characters at
the beginning of a line indicate a comment.

\begin{lstlisting}[style=c++]
CSVReader csv;

csv.trim_fields(true);
csv.set_field_separator('\t');
csv.set_comment("--");
csv.open(infile);
\end{lstlisting}

We can then iterate over the lines of the file as follows, by obtaining the full line as a string or a vector with a string for each field:
\begin{lstlisting}[style=c++]
while (csv.has_next())
{
    vector<string> fields = csv.get_next();
    string line = csv.get_current_raw_line();
    // process the line...
}
csv.close();
\end{lstlisting}

If we know that the lines should have a given number of fields, then we can set this property:
\begin{lstlisting}[style=c++]
csv.set_expected_num_fields(8);
\end{lstlisting}
In this way, lines with a different number of fields will be skipped, and at the end we can check how many (for example to check that no lines were omitted because of errors:
\begin{lstlisting}[style=c++]
csv.skipped_lines(); // returns num. of skipped lines
\end{lstlisting}


\subsection{NameIterator.hpp}

Various functions may need to generate a list of names, for example functions generating new vertices. A \code{NameIterator} takes a prefix and the number of names to generate as input, also guaranteeing that all names have the same length. For example,
\begin{lstlisting}[style=c++]
auto names = NameIterator("U", 11);
for (auto name: names)
{
    cout << name << endl
}
\end{lstlisting}
the code above would print \texttt{U00} to \texttt{U10};


\subsection{Stopwatch.hpp}

A \code{Stopwatch} can be used to compute execution times. We can start it, then call the \code{lap} function multiple times.
Later we can obtain the time passed for each lap, indicating the lap number (starting from 0)
\begin{lstlisting}[style=c++]
Stopwatch clock;
clock.start();
// do something
clock.lap(); // lap1
// do something
clock.lap(); // lap2
clock.millis(0); // milliseconds between start and lap1
clock.sec(0); // seconds between start and lap1
clock.millis(1); // milliseconds between lap1 and lap2
\end{lstlisting}

\subsection{math.hpp and string.hpp}

These modules provides basic mathematical functions not available in the standard library, or not implemented in all the systems where we need our R library to work. \code{math.hpp} declares \code{mean}, \code{stdev}, \code{union}, \code{intersection}, while \code{string.hpp} declares the following functions:

\begin{lstlisting}[style=c++]
string s = "An <example> &";
to_upper_case(s); // "AN <EXAMPLE> &"
to_xml(s); // "AN &lt;EXAMPLE&gt; &amp;"
\end{lstlisting}

\subsection{random.hpp}

Random number generation is another type of functionality that was still not consolidated in the c++ standard library when we started developing our library. The following functions are available:
\begin{lstlisting}[style=c++]
irand(10); // a number (size_t) between 0 and 9 
drand(); // a double between 0 and 1 (excluded)
get_binomial(10); // a size_t between 0 and 10
get_k_uniform(10, 3); // a vector of 3 numbers (size_t) 
                      // between 0 and 9 
\end{lstlisting}

We can also run tests:
\begin{lstlisting}[style=c++]
test(.2); // true with probability .2, false o.w.
vector<double> probs = {.2, .5, .3};
test(probs); // 0 with prob..2, 
             // 1 with prob. 5, 
             // 2 with prob. 3
\end{lstlisting}

\subsection{vector.hpp}

This file declares functions to generate vectors, R style:
\begin{lstlisting}[style=c++]
seq(2, 4) // {2,3,4}
seq(4, 2) // {4,3,2}
\end{lstlisting}
and to update vectors by moving the element from one position to another:
\begin{lstlisting}[style=c++]
vector<size_t> vec = {1,2,3,4,5};
move(vec, 1, 4); // {1,3,5,4,2}
\end{lstlisting}
    

\subsection{hashing.hpp}

This file specifies hash functions for custom types so that they can be used in hash data structures, for example inside an \code{unordered\_map}. One would not normally touch this file, except if there is a need to enable such possibility for types that do not already provide it.

\section{Property Matrix}

Property matrices are sparse data structures where we can associate a value to a structure in a specific context. This is currently used to count motifs (that is, structures: vertices, edges, triangles, \dots) in layers of a multilayer network (that is, contexts) to compute layer comparison functions, and they were defined to support the identification of positions and roles, but in principle property matrices can be used to store adjacency matrices, term-document matrices, etc.

In the following example we define an adjacency matrix:
\begin{lstlisting}[style=c++]
auto A = PropertyMatrix<size_t,size_t,bool>(4,4,false);
A.set(0, 2, true);
A.set(3, 1, true);
A.get(0, 1).value; // false
\end{lstlisting}

We can also specify if a value in not known for a pair structure/context, retrieve the number of unknown values for a context and (for numerical matrices) replace the values with their rank.
\begin{lstlisting}[style=c++]
P.set_na(actor1, layer1);
P.num_na(layer1);
P.rankify();
\end{lstlisting}

\section{Exceptions}

The sub-module \texttt{core/exceptions} contains several exceptions to be thrown by library functions. For example:
\begin{lstlisting}[style=c++]
throw FileNotFoundException(file_name)
\end{lstlisting}
The available exceptions can be found in \code{core/exceptions}. One function to know about is \texttt{assert\_not\_null}, that should be called every time another function takes a pointer as input to check that it is not null. To provide a more useful
error message, \texttt{assert\_not\_null} takes the name of the function and parameter as input. For example,
\begin{lstlisting}[style=c++]
assert_not_null(obj, "VCube::add", "obj")
\end{lstlisting}
is called at the beginning of the method \texttt{add} of the \texttt{VCube} class to check if parameter \texttt{obj} is null.

\section{Observers}

To keep stores synchronized, the library uses the Subject/Observer pattern, with classes contained in \texttt{core/observers}. One specific Observer available in the core module is used to synchronize stores, for example to automatically update a store so that it represents the union of others:
\begin{lstlisting}[style=c++]
ObjectStore<Person> store1;
ObjectStore<Person> store2;
ObjectStore<Person> union_store;

UnionObserver<ObjectStore<Person>, const Person> obs2(&union_store);
store1.attach(&obs2);
store2.attach(&obs2);
\end{lstlisting} 
From now union\_store will represent the union of store1 and store2.
\begin{lstlisting}[style=c++]
store1.add(p1.get()); // now union_store has size 1
store2.add(p1.get());
store2.add(p2.get()); // now union_store has size 2
store1.erase(p1.get());
store2.erase(p1.get()); // now union_store has size 1
\end{lstlisting}
Another usage of observers is to propagate deletions from one store to another.